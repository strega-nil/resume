\documentclass[11pt]{article}

% ================ PACKAGES ====================

\usepackage[reset, a4paper, margin=6mm, top=7mm, right=4mm]{geometry}
% text alignment
\usepackage{ragged2e}
% loremipsum
\usepackage{lipsum}
% text color
\usepackage{color}
\usepackage{xcolor}
% underline
\usepackage{ulem}
\usepackage{contour}
% multiple columns
\usepackage{paracol}
% custom fonts
\usepackage{fontspec}
% paper bg
\usepackage{eso-pic}
% icons
\usepackage{fontawesome5}
% hyperlinks
\usepackage{hyperref}
% better itemize
\usepackage{enumitem}

% ================ SETTINGS ====================

\setlist[itemize]{noitemsep, topsep=-2pt, leftmargin=2ex}

\setromanfont{Georgia}
\setsansfont{Optima}
\setmonofont{Comic Code Ligatures}[Scale=0.8]
\renewcommand{\familydefault}{\rmdefault}

% no page numbers and the sorts
\thispagestyle{empty}
% no indentation at beginning of paragraph
\parindent=0pt
% for underline
\renewcommand{\ULdepth}{3pt}
\renewcommand{\ULthickness}{0.5pt}
\contourlength{2.0pt}

% for links
\hypersetup{
  colorlinks=false,
  urlbordercolor=red,
  pdfborderstyle={/S/U/W 1}
}

% no hyphenation
\tolerance=1
\emergencystretch=\maxdimen
\hyphenpenalty=10000
\hbadness=10000

% ================ MACROS ======================

\newfontfamily\titlethin{Georgia-Regular}
\newfontfamily\titlethick{Georgia-Bold}
\newfontfamily\titlebold{Georgia-Bold}
\newfontfamily\normaltext{Optima}

\newcommand{\resumetitle}[3]{
  \AddToShipoutPictureBG{
    \AtPageUpperLeft {
    \raisebox{-0.09\paperheight}{
      \color{black!85}\rule{2\paperwidth}{\paperheight}}
    }}
  \begin{Center}
    \begingroup
    \titlethin
    \color{black!10}\Huge{#1}
    \titlethick
    \color{black!5}\Huge{#2} \\
    \vspace{2mm}
    \textrm{\color{black!15}\Large{#3}}
    \endgroup
  \end{Center}
  \vspace{7mm}
}

\newcommand{\betteruline}[1]{
  \uline{#1}
}

\newcommand{\sectiontitle}[1]{
  \begingroup
    \titlebold
    \betteruline{\Large\uppercase{#1}  }
    \vspace{1.7mm}
  \endgroup
}

\newcommand{\sectioncontent}[1]{
  \begingroup
    \begin{FlushLeft}
    \vspace{-3mm}
    \sffamily\small#1
    \end{FlushLeft}
  \endgroup
  \vspace{2mm}
}

\newcommand{\job}[3]{
  \begingroup
    \textbf{\small#1} - \small#2
    \hfill\color{black!70}\small{#3}
  \endgroup
}

\newcommand{\project}[2]{
  \begingroup
    \textbf{\small#1}
    \hfill\color{black!70}\small{#2}
  \endgroup
}

\newcommand{\spacevv}{
  \vspace{2mm}
}

\newcommand{\honor}[2]{
  \textcolor{black!70}{#1} - #2 \\
  \vspace{1.5mm}
}

\begin{document}

\resumetitle{Nicole}{Mazzuca} {
  mazzucan@outlook.com |
  (425) 443-3734
}

\columnratio{0.31}
\setlength{\columnsep}{7mm}
\begin{paracol}{2}

\sectiontitle{about me}
\sectioncontent{
  I am a software engineer who has long been fascinated by developer tooling and programming languages.
  I had an early start in the Rust community, then switched over to C++ six years ago.
  I absolutely love developer tooling for its customer-facing nature,
  and for the impact I can make in a customer's every-day experience.
}

\sectiontitle{education}
\sectioncontent{
  \textbf{B.Sc. in Mathematics} \\
  Western Washington University \\
  \textcolor{black!70}{January 2018 - December 2019} \\
  New York University \\
  \textcolor{black!70}{Fall 2016} \\
  \textbf{Associate of Art} \\
  Bellevue College -- Running Start \\
  \textcolor{black!70}{September 2014 - July 2016}
}

\sectiontitle{links}
\sectioncontent{
  \faIcon{github-alt}\hspace{2mm}
  \href{https://github.com/strega-nil}{@strega-nil} \\
  \faIcon{mastodon}\hspace{2mm}
  \href{https://mstdn.social/@streganil}{@streganil} \\
  \faIcon{linkedin-in}\hspace{2.1mm}
  \href{https://www.linkedin.com/in/nicole-mazzuca-18291472/}{Nicole Mazzuca} \\
}

\sectiontitle{skills}
\sectioncontent{
  Expert in Modern C++ \\
  Experienced in Rust \\
  Experienced in CMake \\
  Competent in OCaml \\
  Competent in JavaScript \\
  Competent in Python \\
}

\sectiontitle{achievements}
\sectioncontent{
  Spoke at CppCon 2017 and 2018 \\
  Helped run CSAW 2016 \\
}

\switchcolumn

\sectiontitle{experience}
\sectioncontent{

\job{Microsoft -- Visual C++ Libraries}{Software Engineer}{May 2022 -}
\begin{itemize}
  \item Maintainer of the Microsoft Visual C++ Standard Library
  \begin{itemize}
    \item Code reviewed pull requests from open source contributors and colleagues
    \item Implemented C++ standard papers --
    \href{https://wg21.link/p2494r2}{P2494R2},
    \href{https://wg21.link/p2408r5}{P2408R5},
    \href{https://wg21.link/p2517r1}{P2517R1},
    and \href{https://wg21.link/p2520r0}{P2520R0}
    \item Supported the Windows internal C++ toolset updates
    \item Created a process for community members to add subtitles to our \href{https://github.com/microsoft/STL/wiki/Code-Review-Videos}{Video~Code~Reviews}
  \end{itemize}
  \item Address Sanitizer libraries implementor
  \begin{itemize}
    \item Communicated with internal teams to discover issues and necessary features
    \item Lead Standard Library/Address Sanitizer compatibility efforts -- especially \texttt{std::string} ASan integration
  \end{itemize}
\end{itemize}
\spacevv

\job{Microsoft -- vcpkg}{Software Engineer}{January 2020 - May 2022}
\begin{itemize}
  \item Designed and implemented multiple major features which were blockers for enterprise customers and Visual~Studio integration
  \begin{itemize}
    \item Registries
    \item Manifests
    \item Localization
  \end{itemize}
  \item Lead efforts to open lines of communication with community members
  \begin{itemize}
    \item Added a \#include <C++> channel for vcpkg
    \item Worked to create an agreement for timely PR review
    \item Created the vcpkg RFC process
  \end{itemize}
  \item Lead the macOS pull request testing infrastructure efforts
  \item Code reviewed pull requests from open source contributors and colleagues
\end{itemize}
\spacevv

\job{Microsoft -- vcpkg}{Software Engineering Intern}{July 2019 - August 2019}
\begin{itemize}
  \item Rewrote parts of the filesystem layer
  \item Rewrote the tests to be cross-platform using \href{https://github.com/catchorg/Catch2}{Catch2}
  \item Added benchmarks to the tool
  \item Added hashing code to avoid calling out to platform-specific tools -- Sha1, Sha256, and Sha512
\end{itemize}

}

\sectiontitle{significant projects}
\sectioncontent{

\project{\href{https://includecpp.org}{\#include <C++>}}{2018 -}
\begin{itemize}
  \item Founding member of, and moderator for, the \#include <C++> community -- the largest chatroom for C++ developers.
  \item We have a long-standing commitment for inclusion and diversity in the C++ community.
  \item Additionally, as part of my time in vcpkg, we opened an official vcpkg channel in the \#include <C++> discord.
\end{itemize}
\spacevv

\project{\href{https://learn.microsoft.com/en-us/vcpkg/users/manifests}{vcpkg Manifests}}{May - September 2020}
\begin{itemize}
  \item I lead the charge on designing and implementing the manifests feature from its initial RFC to completion.
  \item Completely changed how users interact with vcpkg,
  while still supporting strict backwards compatibility with existing usecases
  \item The necessary first step towards bringing the C++ package management experience in line with other languages,
  with both package versioning and user-local installation fully depending on manifests.
\end{itemize}
\spacevv
    
\project{\href{https://learn.microsoft.com/en-us/vcpkg/users/registries}{vcpkg Registries}}{September 2020 - March 2021}
\begin{itemize}
  \item Another major feature, necessary for enterprise customers, closed source libraries, and package versioning.
  \item Again, lead the charge on designing and implementing.
  \item Well-loved by the community:
  ``[about registries] Isn't that beautiful. With Conan I couldn't come up with a working setup for this,
  and with vcpkg it just works with a minimal effort.'' (from \href{https://decovar.dev/blog/2022/10/30/cpp-dependencies-with-vcpkg/}{this blog post})
\end{itemize}
\spacevv
}

\normaltext \hfill \tiny Last updated \today

\end{paracol}

\end{document}
